\PassOptionsToPackage{usenames,dvipsnames}{xcolor}

\documentclass[20pt, a0paper, landscape, margin=0mm, innermargin=15mm, blockverticalspace=15mm, colspace=15mm, subcolspace=8mm]{tikzposter}
\usepackage[T1]{fontenc}
\usepackage[utf8]{inputenc}
\usepackage[english]{babel}

\usepackage{amsmath,amsfonts,amssymb,mathtools}

%\usepackage[usenames,dvipsnames]{xcolor}
\usepackage{graphicx,mwe}
\usepackage{filecontents}% http://ctan.org/pkg/filecontents
\usepackage{tikz}
\usepackage{multicol}
\usepackage{adjustbox}
\usepackage{authblk}

\usepackage{enumitem}

\usepackage{caption}
\captionsetup{font=large}

\usepackage{url}
\usepackage[colorlinks=false]{hyperref}
\urlstyle{tt}


%\newcommand*{\doi}[1]{\href{https://doi.org/#1}{\nolinkurl{https://doi.org/#1}}}
\newcommand*{\doi}[1]{\href{https://doi.org/#1}{\nolinkurl{doi:#1}}}


% university colors based on branding guides
\definecolor{OSUorange}{HTML}{C34500}
\definecolor{UConnBlue}{HTML}{000E2F}
\definecolor{GoogleBlue}{HTML}{0266C8}

% \makeatletter
% \def\TP@titlegraphictotitledistance{-4cm}
% \settitle{ \centering \vbox{
% \@titlegraphic \\ [\TP@titlegraphictotitledistance] 
% \centering
% \color{titlefgcolor} {\bfseries \Huge \@title \par} % add \sc for smallcaps
% \vspace*{2em}
% {\huge \@author \par} \vspace*{1em} {\LARGE \@institute}
% }}
% \makeatother

%% Code for increasing tikzfigure caption font size
% \renewenvironment{tikzfigure}[1][]{
%   \def \rememberparameter{#1}
%   \vspace{10pt}
%   \refstepcounter{figurecounter}
%   \centering
%   }{
%     \ifx\rememberparameter\@empty
%     \else %nothing
%     \\[10pt]
%     {\large Fig.~\thefigurecounter: \rememberparameter}
%     \fi
% }


%% Set up a logo on each side
\makeatletter
\newcommand\insertlogoi[2][]{\def\@insertlogoi{\includegraphics[#1]{#2}}}
\newcommand\insertlogoii[2][]{\def\@insertlogoii{\includegraphics[#1]{#2}}}
\newlength\LogoSep
\setlength\LogoSep{-8cm}

\insertlogoi[width=18cm]{OSU-color-horz}
\insertlogoii[width=10cm]{NRG-logo}

\def\TP@titlegraphictotitledistance{-4cm}
\settitle{ \centering \vbox{
%\@titlegraphic \\ [\TP@titlegraphictotitledistance] 
\centering
\color{titlefgcolor} {\bfseries \Huge \@title \par} % add \sc for smallcaps
\vspace*{2em}
{\huge \@author \par} \vspace*{1em} {\LARGE \@institute}
}}

\renewcommand\maketitle[1][]{  % #1 keys
    \normalsize
    \setkeys{title}{#1}
    % Title dummy to get title height
    \node[transparent,inner sep=\TP@titleinnersep, line width=\TP@titlelinewidth, anchor=north, minimum width=\TP@visibletextwidth-2\TP@titleinnersep]
        (TP@title) at ($(0, 0.5\textheight-\TP@titletotopverticalspace)$) {\parbox{\TP@titlewidth-2\TP@titleinnersep}{\TP@maketitle}};
    \draw let \p1 = ($(TP@title.north)-(TP@title.south)$) in node {
        \setlength{\TP@titleheight}{\y1}
        \setlength{\titleheight}{\y1}
        \global\TP@titleheight=\TP@titleheight
        \global\titleheight=\titleheight
    };

    % Compute title position
    \setlength{\titleposleft}{-0.5\titlewidth}
    \setlength{\titleposright}{\titleposleft+\titlewidth}
    \setlength{\titlepostop}{0.5\textheight-\TP@titletotopverticalspace}
    \setlength{\titleposbottom}{\titlepostop-\titleheight}

    % Title style (background)
    \TP@titlestyle

    % Title node
    \node[inner sep=\TP@titleinnersep, line width=\TP@titlelinewidth, anchor=north, minimum width=\TP@visibletextwidth-2\TP@titleinnersep]
        at (0,0.5\textheight-\TP@titletotopverticalspace)
        (title)
        {\parbox{\TP@titlewidth-2\TP@titleinnersep}{\TP@maketitle}};

    \node[inner sep=0pt,anchor=west] 
      at ([xshift=-\LogoSep]title.west)
      {\@insertlogoi};

    \node[inner sep=0pt,anchor=east] 
      at ([xshift=\LogoSep]title.east)
      {\@insertlogoii};

    % Settings for blocks
    \normalsize
    \setlength{\TP@blocktop}{\titleposbottom-\TP@titletoblockverticalspace}
}
\makeatother

\setlength{\columnsep}{2cm}


\title{\parbox{0.7\linewidth}{Swept rule domain decomposition for 1D PDEs on heterogeneous compute clusters}}

\author[1]{Daniel J.\ Magee}
%\institute{Oregon State University}
\author[2]{Kyle E.\ Niemeyer}

\affil[1, 2]{School of Mechanical, Industrial, and Manufacturing Engineering, Oregon State University}

\institute{\href{mailto:mageed@oregonstate.edu}{\nolinkurl{mageed@oregonstate.edu}} and \href{mailto:kyle.niemeyer@oregonstate.edu}{\nolinkurl{kyle.niemeyer@oregonstate.edu}}}

\date{}

\usetitlestyle{Filled}

\begin{document}

\maketitle

\begin{columns} % See Section 4.4
    \column{0.25} % See Section 4.4
    \block{Introduction \& Motivation}{
	Accurate, fast solutions to Partial Differential Equations are essential to simulating physical phenomena with ever increasing fidelity. This greater resolution provides insight that drives technological development, but requires extensive computational resources. Our project applies a new technique (the swept rule) for apportioning these resources (domain decomposition) in PDE solutions to heterogeneous computing clusters.
    \vspace{0.5em}
    
    Recently, high-performance computing clusters, such as ONRL's Titan, have moved hybrid processing paradigm by incorporating accelerating processors like GPUs. The features of GPU architecture originally designed for high resolution and frequency displays, hundreds to thousands of simple, parallel cores, a transparent memory hierarchy, and generous on-chip memory capacity, make it a natural tool for scientific computing.  
    
    \begin{minipage}{0.45\linewidth}
    \centering
    \begin{tikzfigure}
    \includegraphics[width=\linewidth]{BigGpu}
    \end{tikzfigure}
    \captionof{figure}{GPU used in this study: Tesla K40c}
    \end{minipage}
    \hfill
    \begin{minipage}{0.45\linewidth}
    \centering
    \begin{tikzfigure}
    \includegraphics[width=\linewidth]{XEONCPU}
    \end{tikzfigure}
    \vspace{0.25em}
    \captionof{figure}{CPU used in this study: Intel Xeon 2630-E5}
    \end{minipage}
    
    \vspace{1em}
    
    }
    
    \column{0.5}
    \block{Performance results}{
    	
    \begin{minipage}{0.48\linewidth}
    \centering
    \begin{tikzfigure}
    \includegraphics[width=\linewidth]{InitialCmpare}
    \end{tikzfigure}
    \captionof{figure}{Swept Rule compared to Classic applied to heat equation}
    \end{minipage}
	\hfill
	\begin{minipage}{0.48\linewidth}
	\centering
	\vspace{0.25em}
	\begin{tikzfigure}
	\includegraphics[width=\linewidth]{InitialSpeedup}
	\end{tikzfigure}
	\captionof{figure}{Speedup of swept rule compared to classic decomposition.}
	\end{minipage}
    \vspace{1em}
    
    These results are taken from an experiment which varies the node width, gpu affinity, and grid size. The plots here show the best result at each grid size, that is, presuming that the user knows the best run configuration for a given grid size and equation.
    
    \vspace{5.5em}
    
    }
    
    \column{0.25}
    \block{Conclusions}{
	The results of this MPI+CUDA implementation match previously observed trends in CUDA only implementation of the swept rule applied to the Heat equation.
	
	\vspace{0.5em}
	
	\begin{itemize}
		
		\item As communication cost becomes less significant the positive effect of the swept rule diminishes and it's extra overhead hinders it.
		 
		\item Need further investigations on clusters with higher latency and multiple nodes
		
	\end{itemize}

	More equations, particularly the Euler equations for inviscid flow which doesn't work very well on the GPU alone. 
	
	Remember: Upshot philosophy of swept rule -> Do as much work on the data closest to the logic unit before sending and receiving.
	
    }
\end{columns}

% bottom 
\begin{columns}
    \column{0.5}
    \block{Method}{
    \begin{minipage}[t]{0.3\linewidth}
	\centering\textbf{Classic Decomposition}   
    \begin{tikzfigure}
    \includegraphics[width=0.9\linewidth]{ClassicScheme}
    \end{tikzfigure}
    \centering\textbf{Swept Decomposition}
    \begin{tikzfigure}
    \includegraphics[width=0.9\linewidth]{FirstOrderStepOne}
    \end{tikzfigure}
    \begin{tikzfigure}
    \includegraphics[width=0.9\linewidth]{FirstOrderStepTwo}
    \end{tikzfigure}
	\end{minipage}  
	\hfill
	\begin{minipage}[t]{0.67\linewidth} 
	As compute clusters become larger, more processors need to communicate to exchange results to continue a solution. The obvious answer to this challenge is to simply partition the spatial domain and pass the values at the edge of each node (partition) to the neighboring node. We refer to this decomposition method as Classic. 
	
	Classic Decomposition needs to pass every timestep.  Explain why that's a problem briefly.
	
	How swept rule works.  Downside -- extra overhead.
	
	Explicit methods.  Could end with a brief heat equation summary.
	
	GET MORE CITATIONS (QR code for repo?)
	
	Heat EQUATION:
	\begin{equation}
	\frac{\partial T}{\partial t} = \alpha \frac{\partial^2 T}{\partial x^2} \;.
	\end{equation}
	
	Using a first-order, explicit, finite-difference approximation, forward differencing in time and central in space, we can solve for T at the new timestep with:
	\begin{equation}
	T_i^{m+1} = \text{Fo} (T_{i+1}^m + T_{i-1}^m) + (1 - 2 \text{Fo}) T_i^m \;.
	\end{equation}
	where $i$ is the spatial node index and $m$ is the time index corresponding to time $t^m$.

	\end{minipage}
    }

    \column{0.25}
    \block{Works Cited}{
    \begin{enumerate}[leftmargin=1cm,topsep=0pt]
        
        \item DJ Magee \& KE Niemeyer. \texttt{HeterogeneousSwept1D}. GitHub: \url{https://github.com/Niemeyer-Research-Group/HeterogeneousSwept1D}.
        v1.0 (2017)
         
        \item M Alhubail \& Q Wang.
        ``The Swept Rule for Breaking the Latency Barrier in Time Advancing PDEs.'' (2017) \textit{Journal of Computational Physics}, in press.
        \doi{10.1016/j.jcp.2015.11.026}
        
        \item DJ Magee \& KE Niemeyer.
        ``Accelerating solutions of PDEs with GPU-based swept time-space decomposition'' (2017) In preparation, available via {\tt \href{https://arxiv.org/abs/1705.03162}{arXiv:1705.03162}}.
		
    \end{enumerate}

	\vspace{1em}

    }
   
    \column{0.25}
    \block{Acknowledgements}{
    This material is based upon work supported by NASA under award No.~NNX15AU66A under the technical monitoring of Drs.~Eric Nielsen and Mujeeb Malik.
    We also gratefully acknowledge the support of NVIDIA Corporation with the donation of the Tesla K40c GPU used for this research. 
    \\ \\
    \begin{minipage}{0.42\linewidth}
    \begin{tikzfigure}
    \includegraphics[width=\linewidth]{cc-by}
    \end{tikzfigure}
    \end{minipage}
    \hfill
    \begin{minipage}{0.55\linewidth}
    This work is licensed under a Creative Commons Attribution 4.0 International License. To view a copy of this license, visit \url{http://creativecommons.org/licenses/by/4.0/}.
    \end{minipage}
    
    \vspace{2.5em}
%    \innerblock{Image credits}{
%    \small{
%    a: Modified from \url{https://www.olcf.ornl.gov/wp-content/uploads/2011/06/Oefelein2_web.png}, 
%    b: Weber et al., \emph{Combust Flame} (2014) \doi{10.1016/j.combustflame.2014.01.018}
%    c: Intel Skylake CPU \url{http://wimages.vr-zone.net/2015/04/Intel.jpg}
%    d: NVIDIA Tesla K40 \url{http://www.wiredzone.com/mmenglish/Images/actual}
%    e: Intel Xeon Phi \url{https://streamcomputing.eu/wp-content/uploads/2012/11/intel-xeon-phi-card.jpg}
%    }}
    
    \centering\textbf{Further Information}
    \colorlet{innerblocktitlebgcolor}{OSUorange}
    \colorlet{innerblocktitlefgcolor}{white}
    \innerblock{Niemeyer Research Group (OSU)}{\centering{\url{https://niemeyer-research-group.github.io}}}
    
    \vspace{2.5em}
    \colorlet{innerblocktitlebgcolor}{OSUorange}
    \colorlet{innerblocktitlefgcolor}{white}
    \innerblock{Daniel's Site }{\centering{\url{https://www.danieljmagee.com}}}
    
    \vspace{2.5em}
}
    
\end{columns}


\end{document}
